\documentclass[a4paper, 11pt]{article}
\usepackage{graphicx}
\usepackage{amsmath}
\usepackage[pdftex]{hyperref}

% Lengths and indenting
\setlength{\textwidth}{16.5cm}
\setlength{\marginparwidth}{1.5cm}
\setlength{\parindent}{0cm}
\setlength{\parskip}{0.15cm}
\setlength{\textheight}{22cm}
\setlength{\oddsidemargin}{0cm}
\setlength{\evensidemargin}{\oddsidemargin}
\setlength{\topmargin}{0cm}
\setlength{\headheight}{0cm}
\setlength{\headsep}{0cm}

\renewcommand{\familydefault}{\sfdefault}

\title{Introduction to Learning and Intelligent Systems -- Spring 2015}
\author{taubnert@student.ethz.ch\\ junkerp@student.ethz.ch\\ kellersu@student.ethz.ch\\}
\date{\today}

\begin{document}
\maketitle

\section{Project Regression -- Team ``awesome''}

%Briefly describe the steps used to produce the solution. Feel
%free to add plots or screenshots if you think it's necessary. The
%report should contain a maximum of 2 pages.


\subsection{logscore}
Since we can not use logscore directly as a distance function during regression,
we need to transform our data $x,y$.
Suppose we want to search a function $f(x)$.
Then to minimize $logscore(f(x),y)$ we minimize the two-norm $||f'(x) - y'||_2$ instead.
Looking at the definition of $logscore$, we see that this can be accomplished by choosing $f'(x) = \log(1 + f(x))$ and $y' = \log(1 + y)$.
The function $f$ can then be reconstructed by $f(x) = \exp(f'(x)) - 1$.

\subsection{Regressors}
We used a number of different regressors.
Most of them we understand how they work but we also used the \emph{RandomForestRegressor} which we don't understand at all.

In the end, we compared a simple linear regression, a ridge regression, a k-nearest-neighbours regression, a lasso regression with the random forest regression.
We concluded that we can do almost as good as the random forest regression.

\subsection{Features}
Different heuristics lead us to use different basis functions for our features.
Because the data is about train usage and we have a timestamp provided, we assumed that there will be some periodicity observeable.
This assumption let us add the fourier and also the discrete cosine transformation as base-functions.

We deliberately didn't use all of the available data to fit.
We concluded that time in the minute frequency and the weather data D corresponds mostly to noise.

\subsection{Scores}

\begin{tabular}{lll}
Linear        & & \\
Ridge         & & \\
Lasso         & & \\
K-NN          & & \\
Random Forest & &
\end{tabular}

\end{document}
